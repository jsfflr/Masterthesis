\chapter{Introduction} % könnte eigentlich schon fast gestrichen werden

\begin{quote}
The goal of software architecture is to minimize the human resources required to build and maintain the required system.
\end{quote} Robert C. Martin in Clean Architecture


\section{Problem description/Motivation}

\begin{itemize}
	\item Current API can't be reengineered, because of legacy code and high risk of ruining the current functionalities
	\item Clients are coupled to the current old api and to be able to communicate with this service, it has to implement the old api technology 
	\item This is a big effort, because the knowledge of older technologies are not available at newer developer, old technologies relays on older datatypes so that the development time can be increased and also the performance could be worse, because the data package are bigger and the parser for the datatypes are slower (see SOAP vs REST) % TODO: und hier ein paar Quellen einfügen, welche SOAP vs REST zeigen und generell the thesis unterstützen, dass neuere APIs grundsätzlich besser sind (sonst gäbe es ja keinen Grund diese zu entwickeln, wenn es nicht das Ziel zu Verbesserung eines Problems gäbe)
	\item Interoperability is also a bigger topic in the domain of IoT, where heterogeneous protocols have to communicate with each other
	\item We assuming in the problem description that we have a service with an api which is not capable to communicate with a client, because the protocol differs. For that the service has to
\end{itemize}

% Beispiel: Hinweisbox (kurze Tipps oder Zusammenfassungen)
\begin{tcolorbox}[arc=3mm, boxrule=0pt, leftrule=4pt, left=10pt]
Diese Art von Box eignet sich besonders gut, um zwischen sachlichen Abschnitten kurze Zusammenfassungen oder wichtige Hinweise einzufügen – etwa am Kapitelende oder zwischen methodischen Schritten.
\end{tcolorbox}


\section{Goals of the work}

\begin{itemize}
	\item 
\end{itemize}

% Beispiel: Zentrale Infobox (Forschungsfragen, Definitionen, wichtige Aspekte)
\begin{center}
\begin{minipage}{\textwidth}
\begin{tcolorbox}[title=\textbf{Forschungsfragen, Infobox, Definition etc.}, left=1mm, right=1mm]
\raggedright
Diese Box dient der übersichtlichen Darstellung wichtiger Inhalte wie Forschungsfragen, Definitionen oder methodischer Hinweise. Sie kann genutzt werden, um zentrale Aspekte optisch hervorzuheben und den Lesefluss zu unterstützen. \\
\end{tcolorbox}
\end{minipage}
\end{center}

\subsection{Research Questions}


\section{Procedure/Methode}

\section{Limitations/Boundaries}

\begin{itemize}
	\item No transformations on network layer
	\item Just focus on request-response communication protocols - so we don't need to check the behavioural structure of the protocols. We can assume that we just transform between protocols with the same behavioural structure % TODO: what is about api splitting and merging? In this case we also need some kind of state-machines (first call api-1, then api-2, then merge the two responses and send back to client).
	\item Should i solve authorization? % TODO: check which kind of usages are there in an distributed communication (caching, authorization, different datatypes/media-types, ...).
	\item Focus on Web-API's \cite{geewaxAPIDesignPatterns2021} (see \ref{fig:test})
\end{itemize}

\begin{figure}[b]
  \centering
  \includegraphics[width=1.0\textwidth]{../images/web-api-definition.png}
  \caption{Web-API-Definition \cite{geewaxAPIDesignPatterns2021}.}
  \label{fig:test}
\end{figure}

\section{Structure of the work}